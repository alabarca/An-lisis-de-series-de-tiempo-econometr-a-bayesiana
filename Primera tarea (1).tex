\documentclass[12pt,legalpaper]{report}
\usepackage[dvips]{graphicx} \usepackage[ansinew]{inputenc}
\usepackage{euler} \usepackage{epsfig} \usepackage{amssymb}
\usepackage{amscd} \usepackage{amsthm}\usepackage{eucal}
\usepackage{amsmath}
\usepackage[left=0.5cm,top=1cm,right=0.5cm,bottom=0.5cm]{geometry}
%\oddsidemargin -1.5cm \textwidth 19cm \topmargin -2.75cm \leftmargin
%-5cm \textheight 30cm \parindent 2em \parskip 2ex
\newcommand{\est}{\displaystyle} \newcommand{\sen}{\mathop{\rm
sen}\nolimits} \newcommand{\arcsec}{\mathop{\rm arcsec}\nolimits}
\newcommand{\cotan}{\mathop{\rm cotan}\nolimits}
\newcommand{\sgn}{\mathop{\rm sgn}\nolimits}
\newcommand{\dom}{\mathop{\rm Dom}\nolimits}
\newcommand{\codim}{\mathop{\rm codim}\nolimits}
\newcommand{\ran}{\mathop{\rm ran}\nolimits}
\newcommand{\arcsen}{\mathop{\rm arcsen}\nolimits}
\newcommand{\cosec}{\mathop{\rm cosec}\nolimits}
\newcommand{\RR}{\mathbb{R}}
\newcommand{\QQ}{\mathbb{Q}}
\newcommand{\CC}{\mathbb{C}}
\newcommand{\HH}{\mathbb{H}}
\newcommand{\XX}{\mathbb{X}}
\newcommand{\OO}{\mathbb{O}}
\newcommand{\ZZ}{\mathbb{Z}}\newcommand{\lv}{\textbf v}
\newcommand{\lu}{\textbf u}
\newcommand{\lx}{\textbf x}
\newcommand{\ls}{\textbf s}
\newcommand{\ly}{\textbf y}
\newcommand{\Int}{\textbf{Int}}
\newcommand{\Id}{\textrm{id}}
 \newcommand{\pp}{\mathbb{P}}
\newcommand{\mf}{\mathfrak}\newcommand{\mc}{\mathcal}
\renewcommand{\proofname}{Soluci\'on}

\theoremstyle{plain} \newtheorem{teo}{Teorema}[chapter]
\newtheorem{pro}[teo]{Proposici\'on} \newtheorem{lemma}[teo]{Lema}
\newtheorem{corollary}[teo]{Corolario}
\newtheorem{axio}[teo]{Axioma}


\theoremstyle{definition} \newtheorem{Def}[teo]{Definici\'on}
\newtheorem{example}[teo]{Ejemplo}
\pagestyle{empty}
\begin{document}
\begin{figure}[h]\vspace{-0.5cm}
\begin{minipage}{2cm}
\includegraphics[height=1in]{ucv1.eps}
\end{minipage}\hspace{3.5cm}\begin{minipage}{9.5cm}
\begin{center}
\textbf{Universidad Central de Venezuela\\Facultad de
Ciencias\\Postgrado Modelos Aleatorios\\Estad\'istica I}
\end{center}
\end{minipage}\hspace{0.9cm}\begin{minipage}{2cm}
\includegraphics[height=0.7in]{matem.eps}
\end{minipage}
\end{figure}\vspace{-1cm}
\begin{center}
\begin{tabular}{|c|}
  \hline
\vspace{-0.4cm}\\
  % after \\: \hline or \cline{col1-col2} \cline{col3-col4} ...
  \textrm{{\large{$\est\underset{Elaborado\,por\, Alejandro\,\, Labarca}{\textbf{\textrm{Entrega de ejercicios}}}$}}}

  \\\hline
  \end{tabular}
\end{center}

\vspace{1cm}


\noindent\textbf{Pregunta 1.}\,\, Calcular la funci\'on generadora de momentos de una variable aleatoria $X\sim N(\mu,\sigma^{2})$. Considere el caso particular $\mu=0$ y $\sigma^{2}=1$.
\begin{proof}
Como $X\sim N(\mu,\sigma^{2})$, su funci\'on de densidad de probabilidad viene dada por \begin{equation*}
\displaystyle f(x)=\frac{1}{\sigma\sqrt{2\pi}}\exp\left\{\frac{-(x-\mu)^{2}}{2\sigma^{2}}\right\}.
\end{equation*}
N\'otese que, si consideramos el cambio
\begin{equation*}
y=\frac{x-\mu}{\sigma\sqrt{2}}
\end{equation*}
entonces,
\begin{equation*}
dy=\frac{dx}{\sigma\sqrt{2}}.
\end{equation*}
Por consiguiente, utilizando el teorema del cambio de variable,
\begin{equation}\label{1}
\displaystyle\int_{-\infty}^{\infty}f(x)\,dx=
\frac{1}{\sigma\sqrt{2\pi}}\int_{-\infty}^{\infty}\exp\left\{-\frac{(x-\mu)^{2}}{2\sigma^{2}}\right\}\,dx
=\frac{1}{\sqrt{\pi}}\int_{-\infty}^{\infty}e^{-y^{2}}\,dy.
\end{equation}
Veamos ahora que,
\begin{equation*}
\displaystyle\int_{-\infty}^{\infty}e^{-x^{2}}\,dx=1.
\end{equation*}
En efecto, por el teorema de Fubini,
\begin{equation}\label{2}
\left(\displaystyle\int_{-\infty}^{\infty}e^{-x^{2}}\,dx\right)
\left(\displaystyle\int_{-\infty}^{\infty}e^{-y^{2}}\right)
=\int_{-\infty}^{\infty}\left(\int_{-\infty}^{\infty}e^{-y^{2}}\right)e^{-y^{2}}\,dy
=\int_{-\infty}^{\infty}\left(\int_{-\infty}^{\infty}e^{-(x^{2}+y^{2})}\right)\,dy
=\int\int_{\mathbb{R}^{2}}e^{-(x^{2}+y^{2})}\,dx\,dy.
\end{equation}
Ahora, practicamos un cambio de variable en coordenadas polares, esto es,
\begin{equation*}
x=r\cos(\theta),\quad y=r\sin(\theta);\quad \text{con $r>0$ y $\theta\in[0,2\pi)$.}
\end{equation*}
As\'i, por el teorema del Cambio de Variable,
\begin{equation*}
\int\int_{\mathbb{R}^{2}}e^{-(x^{2}+y^{2})}\,dx\,dy.
=\int_{0}^{2\pi}\int_{0}^{\infty}re^{-r^{2}}\,dr\,d\theta
=\pi.
\end{equation*}
Luego, sustituyendo en $(\ref{2})$,
\begin{equation*}
\left(\displaystyle\int_{-\infty}^{\infty}e^{-x^{2}}\,dx\right)^{2}
=\left(\displaystyle\int_{-\infty}^{\infty}e^{-x^{2}}\,dx\right)
\left(\displaystyle\int_{-\infty}^{\infty}e^{-y^{2}}\,dy\right)=\pi.
\end{equation*}
de donde,
\begin{equation*}
\displaystyle\int_{-\infty}^{\infty}e^{-x^{2}}\,dx=\sqrt{\pi}
\end{equation*}
De modo que, sustituyendo en (\ref{1}),
\begin{equation}\label{3}
\displaystyle\int_{-\infty}^{\infty}f(x)\,dx=1.
\end{equation}
Calculemos ahora la funci\'on generadora de momentos de $X$.\\
Dado que,
\begin{equation}\label{4}
M_{X}(t)=\mathbb{E}(e^{tx})=\int_{-\infty}^{\infty}e^{tx}f(x)\,dx
=\frac{1}{\sigma\sqrt{2\pi}}\int_{-\infty}^{\infty}\exp\left\{tx-\frac{(x-\mu)^{2}}{2\sigma^{2}}\right\}\,dx
\end{equation}
donde,
\begin{align*}
tx-\frac{(x-\mu)^{2}}{2\sigma^{2}}
=tx - t^{2}\frac{\sigma^{2}}{2}+t^{2}\frac{\sigma^{2}}{2}+\mu t-\mu t- \frac{(x-\mu)^{2}}{2\sigma^{2}}
&=\mu t +t^{2}\frac{\sigma^{2}}{2}-\left[\mu t+ t^{2}\frac{\sigma^{2}}{2}+ \frac{(x-\mu)^{2}}{2\sigma^{2}} - tx \right]
\\&=\mu t +t^{2}\frac{\sigma^{2}}{2}-\left[\frac{2\sigma^{2}\mu t+t^{2}\sigma^{4}+(x-\mu)^{2}-2\sigma^{2}tx}{2\sigma^{2}}\right]
\\&=\mu t +t^{2}\frac{\sigma^{2}}{2}-\left[\frac{2\sigma^{2}\mu t+t^{2}\sigma^{4}+x^{2}-2x\mu+\mu^{2}-2\sigma^{2}tx}{2\sigma^{2}}\right]
\\&=\mu t +t^{2}\frac{\sigma^{2}}{2}-\left[\frac{x^{2}-2(\mu x+\sigma^{2}tx)+(\mu^{2}+2\sigma^{2}\mu t+t^{2}\sigma^{4})}{2\sigma^{2}}\right]
\\&=\mu t +t^{2}\frac{\sigma^{2}}{2}-\left[\frac{x^{2}-2x(\mu +\sigma^{2}t)+(\mu+\sigma^{2}t)^{2}}{2\sigma^{2}}\right]
\\&=\mu t +t^{2}\frac{\sigma^{2}}{2}-\frac{\left[x-(\mu+\sigma^{2}t)\right]^{2}}{{2\sigma^{2}}}
\end{align*}
Entonces, sustituyendo en (\ref{4})
\begin{align*}
\displaystyle M_{X}(t)&=\frac{1}{\sigma\sqrt{2\pi}} \int_{-\infty}^{\infty}\exp\left\{tx-\frac{(x-\mu)^{2}}{2\sigma^{2}}\right\}\,dx
\\&=\frac{1}{\sigma\sqrt{2\pi}}\int_{-\infty}^{\infty}\exp\left\{\mu t+\frac{t^{2}\sigma^{2}}{2}-\frac{\left[x-(\mu+\sigma^{2}t)\right]^{2}}{{2\sigma^{2}}}\right\}\,dx
\\&=\frac{1}{\sigma\sqrt{2\pi}}\int_{-\infty}^{\infty}\exp\left\{\mu t+\frac{t^{2}\sigma^{2}}{2}\right\}\exp\left\{-\frac{\left[x-(\mu+\sigma^{2}t)\right]^{2}}{{2\sigma^{2}}}\right\}\,dx
\\&=\frac{1}{\sigma\sqrt{2\pi}}\exp\left\{\mu t+\frac{t^{2}\sigma^{2}}{2}\right\}\int_{-\infty}^{\infty}\exp\left\{-\frac{\left[x-(\mu+\sigma^{2}t)\right]^{2}}{{2\sigma^{2}}}\right\}\,dx
\\&{\stackrel{\ref{3}}{=}}\frac{1}{\sigma\sqrt{2\pi}}\exp\left\{\mu t+\frac{t^{2}\sigma^{2}}{2}\right\}
\end{align*}
esto es,
\begin{equation*}
M_{X}(t)=\frac{1}{\sigma\sqrt{2\pi}}\exp\left\{\mu t + \frac{t^{2}\sigma^{2}}{2}\right\}
\end{equation*}
En el caso particular que $\mu=0$ y $\sigma^{2}=1$,
\begin{equation*}
\mu_{X}(t)=\frac{1}{\sqrt{2\pi}}\exp\left\{\frac{1}{2}t^{2}\right\}
\end{equation*}
\end{proof}


\vspace{1cm}
\noindent\textbf{Pregunta 2.}\,\, Si $X_{1},...,X_{n}$ son variables aleatorias independientes con distribuci\'on $N(\mu,\sigma^{2})$, calcular la distribuci\'on $\bar{X}$
\begin{proof}
Sea $X_{1},...,X_{n}$ variables aleatorias independientes con distribuci\'on $N(\mu,\sigma^{2})$.\\
N\'otese que,
\begin{equation*}
\displaystyle \mathrm{M}_{\bar{X}}(t)=\prod_{i=1}^{n}M_{X_{i}}(t/n)=n\mathrm{M}_{X_{1}}(t/n).
\end{equation*}
Por consiguiente,
\begin{equation*}
\mathrm{M}_{\bar{X}}(t)=n\mathrm{M}_{X_{1}}(t/n)=n\exp\left\{\mu\frac{t}{n}+\frac{1}{2}\frac{t^{2}}{n^{2}}\sigma^{2}\right\}
=\exp\left\{n\left(\mu\frac{t}{n}+\frac{1}{2}\frac{t^{2}}{n^{2}}\sigma^{2}\right)\right\}
=\exp\left\{\mu t+\frac{1}{2}\frac{t^{2}}{n}\sigma^{2}\right\}
\end{equation*}
de donde,
\begin{equation*}
\bar{X}\sim N(\mu,\sigma^{2}/n).
\end{equation*}
\end{proof}
\end{document}
